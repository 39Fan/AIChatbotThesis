\documentclass[report]{jlreq}

\begin{document}
	\chapter{序論}
		\section{研究背景}
			現状として、成蹊大学ではAIでも回答できる簡単な質問などが多く教務部に寄せられており、それらの多くが大学のホームページに掲載されている。
			また、年度初めは履修登録などの関係で特にアクセスが集中しており、教務部の負担が大きい。
			現在教務部に問い合わせる方法としては以下のものがある。
			\begin{itemize}
				\item Microsoft Office 365 (Outlook)
				\item 直接教務部まで赴く
				\item 教務部に電話をかける
			\end{itemize}
			これらの方法では知りたいときにすぐにかつ簡単に情報を得られない。
			その中で成蹊大学の情報に詳しいチャットBot導入をすることによって、1次対応をチャットBotが対応することでアクセスの分散が可能になる。
			また、友達感覚で手軽にやり取りができ、何時でも直ぐに回答を得ることができる。
		\section{研究目的}
			AIチャットbotの回答精度の検証として、学生がするであろう質問をリスト化し、それらに対する回答を得る。そして正答率がどれくらいであるのかの統計を取る。

			次にAIチャットBotの回答精度向上のため、目標正答率を90%として質問に正しく回答をできるようにする。
			そして回答に関しては、成蹊大学の情報のみを答えられるようにする。
	\chapter{AIチャットBotの環境構築}
		\section{学習に使用するPDFファイルの一括ダウンロードプログラム}
\end{document}